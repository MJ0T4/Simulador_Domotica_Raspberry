\capitulo{1}{Introducción}

Hoy en día la informática se está introduciendo en nuestra vida cotidiana sin darnos mucha cuenta. Todos los días usamos elementos cotidianos que poco a poco se van informatizando o más bien programando para realizar mejor sus tareas y de manera más sencilla. Debido a este cambio, nació la domótica, que es el conjunto de tecnologías aplicadas al control y la automatización inteligente de la vivienda, que permite una gestión eficiente del uso de la energía, que aporta seguridad y confort, además de comunicación entre el usuario y el sistema \cite{cedom:domotica}.\\
Este es un tema que siempre me ha llamado la atención, el cómo es posible interactuar con objetos a distancia sin que nosotros seamos capaces de poder ver un medio de conexión. Nosotros no somos capaces de ver o notar el bluetooth, o incluso el wifi, pero ahí están, rodeándonos sin que nos demos cuenta.\\ A raíz de todo esto, está poniéndose de moda un nuevo concepto llamado Internet de las Cosas o IoT. El internet de las cosas se refiere a la interconexión digital de objetos cotidianos con Internet, y que actualmente se usa con una denotación de conexión avanzada de dispositivos, sistemas y servicios que va más allá del tradicional M2M (máquina a máquina) y que cubre una amplia variedad de protocolos, dominios y aplicaciones \cite{wiki:iot}.\\
Esto me motivó a intentar conseguir centralizar todos los elementos cotidianos en nuestra mano, es decir, en nuestro smartphone. Por esto, la idea principal de mi trabajo ha sido trabajar sobre estos temas actuales e intentar crear paso a paso un pequeño simulador que nos permita visualizar como podría funcionar una casa inteligente en la realidad y con qué tecnología se podría conseguir.

